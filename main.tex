\documentclass[25pt, margin=1in, innermargin=-1.7in, blockverticalspace=-0.25in, portrait]{tikzposter}
\usepackage{anyfontsize}
\usepackage{uc-theme}

\geometry{paperwidth=36in,paperheight=48in}

% set theme parameters
\tikzposterlatexaffectionproofoff  % turn off tikzposter watermark
\usetheme{UCTheme}
\usecolorstyle{UCStyle}

\title{
	\parbox{\linewidth}{\centering
		ESTABLISHMENT OF VLBI BETWEEN THE GTM AND SEVERAL POSSIBLE  RADIOTELESCOPES: ZELENCHUK, GREEN BANK, PUSHINO, PARKES, AND ALMA \\
		\vspace{1em}
		COSPAR GENERAL ASSEMBLY \\
		MOSCOW, RUSSIA, AUGUST 2014 \\
		\vspace{1em}
	}
}
\author{\fontsize{38}{46}\selectfont
	Victor I. Krilov 
	(Department of Astronomy and Cosmic Geodesy, Moscow State University of Geodesy and Cartography),\\
	Tatiana N. Kokina and Daniel Mendoza-Araiza
	(Center for Astronomy, Autonomous University of Sinaloa UAS),\\
	Julio Saucedo-Morales 
	(Department for Research in Physics, University of Sonora UNISON),\\
	Daniel Flores-Gutierrez 
	(Institute of Astronomy, UNAM).
}

\begin{document}

\maketitle

% Logos
\node[anchor=north west,yshift=-17pt] at (TP@title.north west)
	{\includegraphics[width=0.10\textwidth]{Figures/cospar_moscow_2014.png}};

\node[anchor=west] at (TP@title.west)
	{\includegraphics[width=0.10\textwidth]{Figures/UAS_logo.png}};

\node[anchor=north east,yshift=-17pt] at (TP@title.north east)
	{\includegraphics[width=0.10\textwidth]{Figures/miigaik_logo.png}};

\node[anchor=east] at (TP@title.east)
	{\includegraphics[width=0.10\textwidth]{Figures/UNISON_logo.png}};

% Body text
\begin{columns}

 % First column
\column{0.5}% Width set relative to text width

\block{}{\fontsize{36}{43}\selectfont
Very Large Base Interferometry is a powerful means to determine the Earth's rotational parameters and to do basic research on
lithospheric plaques derive,
marea phenomena,
as well as on the motion of the poles,
among others.
But it can also provide useful data to solve various geophysical problems.
The work presented here is about a problem of both great current interest to Mexican science,
while at the same time it may potentially have a significant impact to the Mexican economy.

The Mexican National Geodetic Network (Red Geodésica Nacional) is linked through NAVSTAR (GPS) to the International System of Coordinates ITRF,
but the presence of seismic activity in most places of Mexico makes it highly desirable to contribute with a point to the ITRF in order to have an even more rigid network for the study of seismic phenomena.
Here we explore the possibility of achieving this with the LMT as a VLBI station.

VLBI is viable in Mexico since the establishment of the GMT (owned by INAOE and UMass) provided a companion radiotelescope can be found.
Here we propose five possible companions:
Green Bank, Krym, Pushino, Parkes, and Alma,
whose positions are shown in ... 
For a preliminary evaluation of these baseline components we use the coordinates given in ...

The principle behind the use of interferometric observations of far away radiosources (quasars) is based on the fact that the quasar's signal do not arrive at the same time to the antennas of the radiotelescopes $Q_1$ and $Q_2$ located at a great distance from each other (baseline).
The time difference is due to the difference in distance from each radiotelescope to the radiosource.

After submitting the abstract to COSPAR we found in: http://www.imfgtm.org,
that the VLBI capability of the GTM had been demonstrated by astronomers from 
INAOE, UMass, Haystack Observatory, MIT and NRAO, 
observing the quasar 1633+382 through VLBI at 3mm, 
using the GTM Alfonso Serrano on the summit of Sierra Negra, Puebla 
in combination with seven radiotelescopes located as shown ...
The fringes detected are shown ...
}

\end{columns}

\end{document}
